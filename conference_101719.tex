\documentclass[conference]{IEEEtran}
\IEEEoverridecommandlockouts
% The preceding line is only needed to  funding in the first footnote. If that is unneeded, please comment it out.
\usepackage{cite}
\usepackage{amsmath,amssymb,amsfonts}
\usepackage{algorithmic}
\usepackage{graphicx}
\usepackage{textcomp}
\usepackage{xcolor}
\def\BibTeX{{\rm B\kern-.05em{\sc i\kern-.025em b}\kern-.08em
    T\kern-.1667em\lower.7ex\hbox{E}\kern-.125emX}}
\begin{document}

\title{Unbalanced Datasets in Text Classification: Impact and Solutions  \\
%{\footnotesize \textsuperscript{*}Note: Sub-titles are not captured in Xplore and
}
%\thanks{Identify applicable funding agency here. If none, delete this.}
%}


\author{\IEEEauthorblockN{1\textsuperscript{st} Given Name Surname}
\IEEEauthorblockA{\textit{dept. name of organization (of Aff.)} \\
\textit{name of organization (of Aff.)}\\
City, Country \\
email address or ORCID}
\and
\IEEEauthorblockN{2\textsuperscript{nd} Given Name Surname}
\IEEEauthorblockA{\textit{dept. name of organization (of Aff.)} \\
\textit{name of organization (of Aff.)}\\
City, Country \\
email address or ORCID}
\and
\IEEEauthorblockN{3\textsuperscript{rd} Given Name Surname}
\IEEEauthorblockA{\textit{dept. name of organization (of Aff.)} \\
\textit{name of organization (of Aff.)}\\
City, Country \\
email address or ORCID}
\and
\IEEEauthorblockN{4\textsuperscript{th} Given Name Surname}
\IEEEauthorblockA{\textit{dept. name of organization (of Aff.)} \\
\textit{name of organization (of Aff.)}\\
City, Country \\
email address or ORCID}
\and
\IEEEauthorblockN{5\textsuperscript{th} Given Name Surname}
\IEEEauthorblockA{\textit{dept. name of organization (of Aff.)} \\
\textit{name of organization (of Aff.)}\\
City, Country \\
email address or ORCID}
\and
\IEEEauthorblockN{6\textsuperscript{th} Given Name Surname}
\IEEEauthorblockA{\textit{dept. name of organization (of Aff.)} \\
\textit{name of organization (of Aff.)}\\
City, Country \\
email address or ORCID}
}

\maketitle

\begin{abstract}

\end{abstract}

\begin{IEEEkeywords}
sentiment Analysis, Roots, GloVe, word vector representations , Machine learning
\end{IEEEkeywords}

\section{Introduction}
One of the known issues in natural language processing is assigning text to classes dubbed "Text Classification."
%% talk about features
It's one of the most active research areas, especially with the advent of social media platforms, where a massive quantity of texts in many domains are generated in a matter of seconds.Sentiment analysis is a part of Text Classification, it aims to forecast and categorize feelings expressed in comments or tweets into three categories: positive, negative, and neutral. 

In the literature, many strategies have been used to address this issue. Deep learning and related algorithms are becoming increasingly popular. The act of preprocessing is crucial to the creation of models. To feed machine learning algorithms, words and texts must be vectorized.
 
 The major difficulty of this kind of classification is the unbalanced data ........
 %% talk about umballancing problem , what is it why it occurs  (5 lines)
 
 
An appropriate dataset is required for creating performant deep learning models. In sentiment analysis, there is a major exception: the neutral class dominates the comments and tweets, resulting in datasets that are practically unbalanced. Different approaches were employed in recent studies to overcome this drawback. 

Approaches such as oversampling, undersampling, and synthetic balance have been used. Everyone has advantages and disadvantages. We compare these strategies in text categorization, particularly in sentiment analysis, where we apply a prominent preprocessing methodology and the deep learning BERT algorithm, in our study. The F1-score and recall measures are used to evaluate our system. 
%% detail the approaches procedures ( ) lines)


%%What you gonna do in this paper???? (5 lines)



The remainder of the paper is structured as follows:
The second section of the paper goes over the existing literature.
Section 3 discusses the strategies for balancing datasets and the deep learning algorithm used to extract polarity from tweets.
We look about the proposed approach and methodology for extracting polarity from tweets in Section 4.
Finally, we offer the results of the system's evaluation on an unbalanced dataset, as well as some concluding observations and perspectives. 


\section{Related Work}


\subsection{Results and Discussion:}\label{AA}


\section{Conclusion}

\bibliographystyle{IEEEtran}


%\bibliography{IEEEabrv,../references/ref01}

\end{document}
